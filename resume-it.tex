\documentclass[letterpaper,11pt]{article}

\usepackage{latexsym}
\usepackage{titlesec}
\usepackage{marvosym}
\usepackage[usenames,dvipsnames]{color}
\usepackage{verbatim}
\usepackage{enumitem}
\usepackage[hidelinks]{hyperref}
\usepackage{fancyhdr}
\usepackage[italian]{babel}
\usepackage{tabularx}
\input{glyphtounicode}

\pagestyle{fancy}
\fancyhf{}
\fancyfoot{}
\renewcommand{\headrulewidth}{0pt}
\renewcommand{\footrulewidth}{0pt}

\addtolength{\oddsidemargin}{-0.5in}
\addtolength{\evensidemargin}{-0.5in}
\addtolength{\textwidth}{1in}
\addtolength{\topmargin}{-.5in}
\addtolength{\textheight}{1.0in}

\urlstyle{same}

\raggedbottom
\raggedright
\setlength{\tabcolsep}{0in}

\titleformat{\section}{
	\vspace{-4pt}\scshape\raggedright\large
}{}{0em}{}[\color{black}\titlerule \vspace{-5pt}]

\pdfgentounicode=1

\newcommand{\resumeItem}[1]{
	\item\small{
		{#1 \vspace{-2pt}}
	}
}

\newcommand{\resumeSubheading}[4]{
	\vspace{-2pt}\item
	\begin{tabular*}{0.97\textwidth}[t]{l@{\extracolsep{\fill}}r}
		\textbf{#1} & #2 \\
		\textit{\small#3} & \textit{\small #4} \\
	\end{tabular*}\vspace{-7pt}
}

\newcommand{\resumeSubSubheading}[2]{
	\item
	\begin{tabular*}{0.97\textwidth}{l@{\extracolsep{\fill}}r}
		\textit{\small#1} & \textit{\small #2} \\
	\end{tabular*}\vspace{-7pt}
}

\newcommand{\resumeProjectHeading}[2]{
	\item
	\begin{tabular*}{0.97\textwidth}{l@{\extracolsep{\fill}}r}
		\small#1 & #2 \\
	\end{tabular*}\vspace{-7pt}
}

\newcommand{\resumeSubItem}[1]{\resumeItem{#1}\vspace{-4pt}}

\renewcommand\labelitemii{$\vcenter{\hbox{\tiny$\bullet$}}$}

\newcommand{\resumeSubHeadingListStart}{\begin{itemize}[leftmargin=0.15in, label={}]}
\newcommand{\resumeSubHeadingListEnd}{\end{itemize}}
\newcommand{\resumeItemListStart}{\begin{itemize}}
\newcommand{\resumeItemListEnd}{\end{itemize}\vspace{-5pt}}

\begin{document}

	\begin{center}
		\textbf{\Huge \scshape Claudio Genovese} \\ \vspace{8pt}
		\href{mailto:genovese.claudio997@gmail.com}{\underline{genovese.claudio997@gmail.com}} $|$ 
		\href{https://linkedin.com/in/claudiogenovese}{\underline{linkedin.com/in/claudiogenovese}} $|$
		\href{https://github.com/g3ncl}{\underline{github.com/g3ncl}} $|$ 
		\href{https://g3n.cl}{\underline{g3n.cl}}
	\end{center}

	\section{Esperienza}
	\resumeSubHeadingListStart
	
	\resumeSubheading
	{Software Engineer}{Dic 2022 -- Presente}
	{NTT DATA Italia, Customer Experience}{Milano, Italia}
	
	\resumeItemListStart
	\resumeItem{Sviluppo software Fullstack: Esperto in Java e SQL per applicazioni backend robuste, estendo senza soluzione di continuità la mia competenza al frontend. Utilizzando React e Next.js, creo interfacce utente dinamiche e reattive con un focus sulla modularità e integrazione senza soluzione di continuità. Questo approccio full-stack mi consente di creare soluzioni end-to-end che sono efficienti e user-friendly.}
	\resumeItem{Risoluzione proattiva dei problemi: Adottando una posizione proattiva nella risoluzione dei problemi, identifico attivamente e affronto le sfide tecniche durante tutto il ciclo di sviluppo. Il mio coinvolgimento nella risoluzione dei bug, nel troubleshooting e nell'ottimizzazione delle prestazioni garantisce la consegna di soluzioni software robuste ed efficienti.}
	\resumeItem{Documentazione completa: La creazione meticolosa di documentazione tecnica e funzionale dettagliata rappresenta una pietra angolare delle mie responsabilità. Questa documentazione non solo chiarisce l'architettura del software e le decisioni di design, ma rappresenta anche una risorsa preziosa sia per gli stakeholder interni che per gli utenti finali.}
	\resumeItemListEnd
	
	\resumeSubHeadingListEnd

	\section{Istruzione}
	\resumeSubHeadingListStart
	
	\resumeSubheading
	{Laurea in Ingegneria Informatica (non completata)}{Set 2016 -- Ago 2022}
	{Politecnico di Milano}{Milano, Italia}
	\resumeItemListStart
	\resumeItem{Gli esami sostenuti includono: Database, Reti, Sistemi Informativi, Architettura di Computer e Sistemi Operativi, Fondamenti di Informatica, Logica e Algebra}
	\resumeItemListEnd
	
	\vspace{4pt}
	\resumeSubheading
	{Diploma di Liceo Scientifico}{Set 2011 -- Lug 2016}
	{Liceo Scientifico Galileo Galilei}{Potenza, Italia}
	\resumeItemListStart
	\resumeItem{Voto finale: 78/100}
	\resumeItemListEnd
	
	\resumeSubHeadingListEnd

	\section{Competenze Tecniche}
	\begin{itemize}[leftmargin=0.15in, label={}]
		\small{\item{
				\textbf{Linguaggi}{: Javascript, Typescript, Java, Python, SQL, HTML/CSS3} \\
				\textbf{Framework}{: Next.js, Bulma} \\
				\textbf{Strumenti di Sviluppo}{: Git, Docker, VS Code, PyCharm} \\
				\textbf{Librerie}{: React, Material-UI, pandas, NumPy, Matplotlib, Selenium}
		}}
	\end{itemize}

\end{document}

